%%%%%%%%%%%%%%%%%%%%%%%%%%%%%%%%%%%%%%%%%
% "ModernCV" CV and Cover Letter
% LaTeX Template
% Version 1.1 (9/12/12)
%
% This template has been downloaded from:
% http://www.LaTeXTemplates.com
%
% Original author:
% Xavier Danaux (xdanaux@gmail.com)
%
% License:
% CC BY-NC-SA 3.0 (http://creativecommons.org/licenses/by-nc-sa/3.0/)
%
% Important note:
% This template requires the moderncv.cls and .sty files to be in the same
% directory as this .tex file. These files provide the resume style and themes
% used for structuring the document.
%
%%%%%%%%%%%%%%%%%%%%%%%%%%%%%%%%%%%%%%%%%

%----------------------------------------------------------------------------------------
%	PACKAGES AND OTHER DOCUMENT CONFIGURATIONS
%----------------------------------------------------------------------------------------

\documentclass[11pt,a4paper,sans]{moderncv} % Font sizes: 10, 11, or 12; paper sizes: a4paper, letterpaper, a5paper, legalpaper, executivepaper or landscape; font families: sans or roman

\moderncvstyle{casual} % CV theme - options include: 'casual' (default), 'classic', 'oldstyle' and 'banking'
\moderncvcolor{orange} % CV color - options include: 'blue' (default), 'orange', 'green', 'red', 'purple', 'grey' and 'black'

\usepackage[scale=0.75]{geometry} % Reduce document margins
%\setlength{\hintscolumnwidth}{3cm} % Uncomment to change the width of the dates column
%\setlength{\makecvtitlenamewidth}{10cm} % For the 'classic' style, uncomment to adjust the width of the space allocated to your name

%----------------------------------------------------------------------------------------
%	NAME AND CONTACT INFORMATION SECTION
%----------------------------------------------------------------------------------------

\firstname{Salvatore} % Your first name
\familyname{Tomaselli} % Your last name

% All information in this block is optional, comment out any lines you don't need
\title{Curriculum Vitae}
% \address{Via Martiri d'Ungheria, 60}{Nicolosi, Italy 95030}
\address{Batterigatan 6B lgh 1102}{412 76 G\"{o}teborg}
% \mobile{+39 3299588380}
% \phone{+46 (0)738710132}
% \fax{(000) 111 1113}
\email{tiposchi@tiscali.it}
\homepage{http://ltworf.github.io/}{http://ltworf.github.io/} % The first argument is the url for the clickable link, the second argument is the url displayed in the template - this allows special characters to be displayed such as the tilde in this example
% \extrainfo{http://github.com/ltworf/}
% \extrainfo{additional information}
\photo[70pt][0.4pt]{pictures/elegante} % The first bracket is the picture height, the second is the thickness of the frame around the picture (0pt for no frame)
%\quote{"A witty and playful quotation" - John Smith}

%----------------------------------------------------------------------------------------

\begin{document}

\makecvtitle % Print the CV title

%----------------------------------------------------------------------------------------
%	EDUCATION SECTION
%----------------------------------------------------------------------------------------

\section{Education}

\cventry{2011--2013}{Msc Computer Systems and Networks}{Chalmers tekniska h\"{o}gskola}{G\"{o}teborg, Sweden}{\textit{GPA 4.2/5}}{}
\cventry{2007--2011}{B.S. Computer science}{Universit\`{a} degli studi di Catania}{Catania, Italy}{\textit{GPA 110/110} with honors}{}  % Arguments not required can be left empty

\section{Publications}
\cvitem{Title}{Towards Lightweight Logging and Replay of Embedded, Distributed Systems}
\cvitem{Authors}{Salvatore Tomaselli, Olaf Landsiedel}
\cvitem{Description}{In this paper we introduce MILD; providing Minimal Intrusive Logging and Deterministic replay.}
\cvitem{In proceedings}{ASCoMS 2013}

\noindent\hfil\rule{0.7\textwidth}{.4pt}\hfil % Separator as horizontal line

\cvitem{Title}{LibReplay: Deterministic Replay for Bug Hunting in Sensor Networks}
\cvitem{Authors}{Olaf Landsiedel, Salvatore Tomaselli, Elad Schiller}
\cvitem{In proceedings}{\href{a href="https://www.cister.isep.ipp.pt/ewsn2015/}{EWSN'15}}

\section{Masters Thesis}

\cvitem{Title}{Debugging Wireless Sensor Networks with Incremental Snapshots}
\cvitem{Supervisor}{Professor Olaf Landsiedel}
\cvitem{Description}{This thesis intoduced a tool to capture event from a wireless sensor network, where nodes use TinyOS, and allow deterministic replay of the network inside a simulator.}

\section{Bachelor Thesis}
\cvitem{Title}{Sviluppo di server web e sistema di caching per contenuti dinamici}
\cvitem{Supervisor}{Professor Giuseppe Pappalardo}
\cvitem{Description}{This thesis was focused on the implementation of an HTTP/DAV server that can cache and invalidate server-generated responses, leading to greatly reduced IO and CPU loads.}

\section{Skills}

\cvitem{Languages}{ In order of experience:
    Python, C, C++, nesC, Java, JavaScript, Haskell, PHP
    }
\cvitem{Databases}{PostgreSQL, MySQL, MongoDB, Redis, Elasticsearch
    }

\cvitem{Frameworks}{Django, Qt5, Flask}
\cvitem{Software}{Debian, GNU Autotools, Git, SaltStack, Jenkins, unittest}
\cvitem{Technical}{
    Experience in making C code work on various POSIX platforms; using pthreads, atomic operations, networking.
    \newline
    Programming on embedded systems, asynchronous or event-based.
    }
\section{Notable free software projects}
\cvitem{typedload}{
    A library to load json data into Python typed data structures.
    \newline
    It is used to enable mypy to statically check code which loads data into
    dynamic datatypes (list, dictionary) making sure it conforms to certain
    static types instead.
    \newline
    \url{https://github.com/ltworf/typedload}
    }
\cvitem{Relational}{
    Relational, a relational algebra parser/compiler/optimizer, used in academia and as educational tool. Implemented in Python
    \newline
    Can compile relational algebra to Python, and perform some optimizations on the parse tree.
    It provides a GUI for ease of use.
    \newline
    \url{http://ltworf.github.io/relational/}
    }
\cvitem{lapdog}{
    C++ daemon that uses ICMP+ARP to generate events upon appearing/disappearing of network devices.
    \newline
    \url{https://github.com/ltworf/lapdog}
    }
\cvitem{weborf}{
    A HTTP/DAV server, implemented in C.
    \newline
    \url{http://ltworf.github.io/weborf/}
    }
\cvitem{Canary}{
    Dynamic library to monitor for heap overflows, implemented in C.
    \newline
    \url{https://github.com/ltworf/canary}
    }

\section{Experience}

\cventry{2016-}
    {Software developer}
    {Cyxtera}
    {G\"{o}teborg, Sweden}
    {}
    {
        The product I work on, is a VPN solution that comes with advanced
        permissions. The solution includes clients, and an appliance.
        \newline
        The product runs on a custom Ubuntu based distribution, on which I work
        on. There is a daemon in charge of converting user settings to
        configuration and I work on that too.
        \newline
        I also do general back-end work for the internal testing solutions
        and for the linux clients.
    }

\cventry{2014-2015}
    {C++ Software Engineer}
    {Minerva Networks}
    {S. G. La Punta, Italy}
    {}
    {
        The company, US based, sells software and hardware solutions to content providers, to provide live programs or download on demand.
        \newline
        The C++ team provides a scalable and geolocated version of the backend API.
        \newline
        My role included the integration of Elasticsearch within the API, to provide the devices with full text search and filtering.
        I also contributed to solutions to deploy and manage clusters.
    }

\cventry{2013-2014}
    {Backend Python Developer}
    {Duego}
    {G\"{o}teborg, Sweden}
    {}
    {
        The position involved writing and maintaining a REST API in Python for the clients to use, and managing the servers (production and staging).
        \newline
        The API code was based upon different technologies: Flask, Redis, Elasticsearch, Mongodb.
        For the staging and production environment I had to work with AWS, autoscaling, Jenkins, Saltstack, graphite, New Relic and several tools that were developed in house.
    }


\section{Contributed to}
\cvitem{Subsurface}{The only divelog available on Linux systems.
    \newline
    I wrote a functionality to export the logs on a website I made and some other contributions and translations to Italian.}
\cvitem{Debian}{
    At the present time I am a Debian Maintainer (DM).
    \newline
    I maintain mostly packages in C and Python; the most used of them is xinetd.}

\section{Languages}

\cvitemwithcomment{Italian}{Mothertongue}{}
\cvitemwithcomment{English}{Fluent}{}
\cvitemwithcomment{French}{Basic}{}
\cvitemwithcomment{Swedish}{Very basic}{}


\section{Interests}
\cvlistdoubleitem{Bass guitar}{Board games}
\cvlistdoubleitem{SCUBA diving}{Literature}
\cvlistitem{Electric guitar}

\section{Other}
\cvitem{}{I am physically disabled, and I am mentioning it because some countries have specific legislation.}


%----------------------------------------------------------------------------------------
%	COVER LETTER
%----------------------------------------------------------------------------------------

% To remove the cover letter, comment out this entire block

% \clearpage
%
% \recipient{HR Departmnet}{Corporation\\123 Pleasant Lane\\12345 City, State} % Letter recipient
% \date{\today} % Letter date
% \opening{Dear Sir or Madam,} % Opening greeting
% \closing{Sincerely yours,} % Closing phrase
% \enclosure[Attached]{curriculum vit\ae{}} % List of enclosed documents
%
% \makelettertitle % Print letter title
%
% \lipsum[1-3] % Dummy text
%
% \makeletterclosing % Print letter signature

%----------------------------------------------------------------------------------------

\end{document}
